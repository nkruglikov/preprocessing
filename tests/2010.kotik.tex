\documentclass[twoside]{article}
\usepackage{iip8}
\NOREVIEWERNOTES

\begin{document}
\title
    {Скелетизация полутоновых изображений отпечатков пальцев для~задачи идентификации личности}
\author
    {Котик~С.\,В.}
\email
    {kot\_serg@inbox.ru}
\organization
    {Москва, ВЦ РАН}
\thanks
    {Работа выполнена при финансовой поддержке РФФИ, проект \No\,08-07-00338.}
\abstract
    {В статье рассматривается обработка изображений отпечатков пальцев для последующего распознавания по особым точкам папиллярного узора. Для этой  цели строится скелет полутонового изображения. Основной принцип - построение скелета полутонового изображения из частей скелетов бинарных изображений, полученных в результате бинаризации исходной картинки с различными порогами. Такая совокупность фрагментов скелета позволяет получить достаточно четкое бинарное изображение отпечатка пальца. По~нему в свою очередь строится  окончательный скелет, позволяющий проводить сравнение по особым точкам, который и рассматривается как скелет серого изображения.}
\titleEng
    {Grayscale images of fingertips prints skeletonization concerning the~problem of personal identification}
\authorEng
    {Kotik~S.}
\organizationEng
    {CCAS, Moscow, Russia}
\abstractEng
    {In this article fingertips prints images processing is considered for the purpose of further recognition by particular points of a finger papillary pattern. Grayscale image skeleton is constructed. The main idea is construction of the grayscale image skeleton using some parts of binary images skeletons. These binary images one can obtain after binarization of initial image with various brightness levels. Such skeleton fragments collection makes rather accurate fingertip binary image. The next step is constructing of the final skeleton of this binary image, which one can call the grayscale image skeleton. By means of this final skeleton one can make recognition by particular points of a finger papillary pattern.}
\maketitle

Задача распознавания личности по биометрической информации становится все более актуальной в наше время. И не только в криминалистике "--- в~повседневной жизни на режимных объектах или в крупных корпорациях со строгой политикой безопасности зачастую требуется ограничить доступ <<только для своих>> в какие-то помещения, либо к~важной информации. По~радужной оболочке глаза, по голосу, по двумерному и по объемному изображению можно идентифицировать человека. Но~одним из наиболее распространенных и надежных способов было и остается  распознавание по отпечатку пальца.

Проблемы здесь, как и в распознавании по другим биометрическим признакам кроются в <<некачественных>> признаках "--- смазанных или нечетких отпечатках. Самый распространенный метод сравнения отпечатков "--- по особым точкам папиллярного узора "--- очень требователен к качеству полученного изображения. Обработка полутонового изображения и построение скелета папиллярного узора позволяет получить достаточно четкое бинарное изображение на основе исходной <<некачественной>> картинки.

Скелетизация серых (полутоновых) изображений "--- очень интересная проблема; не существует четкого определения данного понятия и детально проработанной теоретической базы. Однако на данный момент в мире разработано несколько подходов, которые можно условно разделить на две группы. В первой группе изображение рассматривается как трехмерная поверхность, где два измерения "--- координаты пикселя, а третье "--- его яркость. Во второй группе алгоритмов производится итерационное удаление точек изображения, пока не останутся срединные оси толщиной в один пиксель (они~и~будут являться скелетом).

В предлагаемом в данной статье методе скелет полутонового изображения будет рассматриваться как составной из частей скелетов бинарных изображений, и, если рассматривать изображение как тремерную поверхность, линия скелета будет являться <<гребнем хребта>>.

Рассматривается алгоритм построения скелета серого изображения на основе бинаризации с~различными порогами исходного изображения для получения четкой бинарной картинки, позволяющей проводить распознавание отпечатков пальцев по~особым точкам.

\section{Некоторые сведения о скелетизации бинарного растрового изображения}
Скелетом плоской области называется множество ее внутренних точек, имеющих не менее двух ближайших граничных точек.

Другое название скелета "--- серединные оси или симметрические оси области. С каждой точкой скелета связана радиальная функция или ширина, которая задает для этой точки ее расстояние от границы области. На основании скелета и ширины можно однозначно реконструировать саму замкнутую область как объединение всех кругов с центрами в точках скелета и радиусами, задаваемыми радиальной функцией. Поэтому скелет вместе с радиальной функцией является, фактически, одним из способов представления замкнутой области.

Скелетом бинарного изображения будем называть объединение всех скелетов компонент изображения. Та часть скелета изображения, которая относится к черным компонентам, составляющим фигуру, называется внутренним скелетом, а часть, относящаяся к белым фоновым компонентам "--- внешним скелетом бинарного изображения.

Пустым кругом будем называть круг, не имеющий внутренних точек, которые являются граничными для непрерывной фигуры. Максимальным пустым кругом называется пустой круг, не содержащийся целиком ни в каком другом пустом круге, рис.\,\ref{pic1}. Каждый круг с центром в точке скелета и радиусом, равным радиальной функции в этой точке, является пустым. Такой круг имеет по крайней мере две граничные точки, общие с границей фигуры, поэтому он является максимальным.

Это свойство является характеристическим для точек, составляющих скелет изображения: скелет является геометрическим местом точек "--- центров максимальных пустых кругов.

\begin{figure}[t]
  \centering
   \includegraphics[width=7cm]{Kotik_pic4.eps}
  \caption{Скелет изображения и максимальный пустой круг.}
  \label{pic1}
\end{figure}

В качестве наиболее простого случая рассмотрим пример фигуры, границей которой является простой многоугольник, рис.\,\ref{pic2}. Максимальные пустые круги касаются границы по крайней мере в~двух точках. Точка касания может быть вершиной многоугольника, либо быть внутренней точкой какого-либо отрезка прямой, являющегося стороной многоугольника. От типа точек касания (вершины или нет)  зависит форма серединных осей, составляющих скелет.

Представим многоугольную границу как объединение непересекающихся множеств точек "--- так называемых сайтов. Сайтами будем считать вершины многоугольников и связные множества остальных точек, т.\,е. стороны многоугольников без вершин (открытые сегменты). Таким образом, имеет сайты-точки и сайты-сегменты.

Пусть максимальный пустой круг касается границы ровно в двух точках. Эти точки принадлежат соответствующим сайтам. Возможны три комбинации типов сайтов: два сайта-сегмента, два сайта-точки, сайт-точка и сайт-сегмент.

В окрестности точки "--- центра пустого круга найдутся другие точки, которые также равноудалены от пары этих
же сайтов. Следовательно, в окрестности рассматриваемой точки "--- центра пустого круга "--- серединная ось представляет собой линию, равноудаленную от пары сайтов. В зависимости от типажа этих сайтов форма линии будет либо прямая, либо парабола.

\begin{figure}[t]
  \centering
   \includegraphics[width=7cm]{Kotik_pic5.eps}
  \caption{Скелет простого многоугольника}
  \label{pic2}
\end{figure}

В скелете также выделяются точки соединения трех или большего числа серединных осей. Каждая такая точка является центром пустого круга, касающегося границы в трех или более точках. Кроме того, существуют еще такие точки скелета, в которых соединяются всего две оси, одна из которых является параболой, а другая "--- либо прямая, либо тоже парабола. У кругов с центрами на таких линиях есть общий сайт-точка.
Вершинами степени 1 являются концевые точки скелета, из которых выходит только одно ребро, вершинами степени 2 "--- такие, из которых выходят два ребра и вершинами степени 3 "--- такие, из которых выходят три ребра.

Из всего вышесказанного следует, что скелет многоугольника представляет собой плоский граф~\cite{bib1,bib2}.

\section{Общий подход}
На вход алгоритму подаётся полутоновое изображение отпечатка пальца, на выходе получаем бинарное изображение скелета, по сути бинаризованную и <<почищенную>> исходную картинку. Алгоритм относится к первому классу из упомянутых во вступлении "--- изображение рассматривается как трехмерная поверхность. Основная идея состоит в~том, что скелет серого изображения собирается на~основе частей скелетов бинарных растровых <<срезов>>, полученных при простой бинаризации исходного изображения по нескольким уровням яркости. Идея обработки полутонового изображения на основе полученных бинарных <<срезов>> уже предлагалась ранее в работе \cite{bib3}.

\begin{figure}[t]
    \centering
    \includegraphics[width=\linewidth]{Kotik_pic1.eps}
    \caption{Исходное полутоновое  изображение
    $\to$ бинарные <<срезы>>
    $\to$ фрагменты бинарных изображений, восстановленные на основе <<почищенных>> скелетов
    $\to$ бинарное изображение, построенное на основе полученного скелета серого изображения.}
    \label{pic3}
%\end{figure}
%\begin{figure}[t]
    \medskip
    ~\hskip3mm
    \includegraphics[width=\linewidth]{Kotik_pic2.eps}
    \caption{Бинарный <<срез>> $\to$ то же изображение после первого этапа $\to$ то же изображение после второго \mbox{этапа}.}
    \label{pic4}
\end{figure}

Количество <<слоев>> и пороговые значения для бинаризации позволяют добиваться нужного соотношения качество-скорость. По алгоритму, описанному ниже для каждого скелета бинарного растрового среза производится отбор ветвей и вершин, так называемая <<чистка>>, после чего по отобранным ветвям и вершинам производится восстановление фрагмента бинарного изображения. Следующим шагом такие фрагменты накладываются друг на друга, и получается конечное четкое бинарное изображение отпечатка пальца, по которому уже и строится скелет, позволяющий проводить сравнение особых точек папиллярного узора. На~рис.\,\ref{pic3} показаны основные шаги алгоритма. Исходное серое изображение отпечатка пальца берется инвертированным для удобства обработки. В общих чертах предложенный подход рассматривался в работе~\cite{bib5}.

\begin{figure}[t]
    \centering
    \includegraphics[width=7cm]{Kotik_pic3.eps}
    \caption{Исходные серые изображения и получившиеся итоговые бинарные изображения, по которым строится скелет.}
    \label{pic5}
\end{figure}

\section{Алгоритм <<чистки>> скелетов\\ бинарных изображений}
Алгоритм состоит из двух этапов, рис.\,\ref{pic4}. После первого этапа остаются только те ветви скелета, обе вершины которых удовлетворяют пороговым условиям для радиуса максимального пустого круга. Представление о подмножестве исследуемых изображений (отпечатки пальцев) позволяет нам ввести такой фильтр, и, таким образом, мы избавляемся от слишком толстых и слишком тонких линий, которые заведомо не являются участками папиллярного узора.

Далее мы избавляемся от шумов, появившихся при бинаризации, вводя порог для длины цепи вершин и ветвей скелета. Цепь "--- все последовательные ветви и вершины скелета, конечной вершиной цепи может служить  вершина скелета степени 1, либо степени 3. На данном этапе мы избавляемся от многочисленных <<перемычек>> между папиллярами, сильно зашумляющих изображение.

На обоих этапах <<чистки>> принимаются во внимание особенности изображений отпечатков пальцев, что позволяет оставлять только те фрагменты скелета, которые имеют отношение к значимой части изображения. В дальнейшем планируется разработать адаптивную <<чистку>> шумов для более аккуратной обработки особых точек папиллярного узора.

\section{Исходные и обработанные\\ изображения}
На рис.\,\ref{pic5} показаны исходные серые изображения и соответствующие им конечные бинарные изображения, по которым строятся итоговые серые скелеты.

Как видно, по некачественному изображению не~удалось восстановить полностью четкую бинарную картинку, однако достаточно много особых точек на картинке присутствуют. На более четких исходных изображениях строится почти идеальная картинка.

\balance
\section{Выводы}
Предлагаемый метод скелетизации полутонового изображения разработан для определенного подмножества изображений "--- отсканированных отпечатков пальцев. Он позволяет на основе исходной, зачастую,  <<некачественной>> картинки получить пригодное для дальнейшего распознавания бинарное изображение папиллярного узора. \mbox{По сути}, это~некий метод интеллектуальной бинаризации серого изображения на основе скелетизации. В ходе дальнейших исследований планируется разработка новых фильтров на этапе <<чистки>> скелета бинарного изображения, адаптивная обработка особых точек папиллярного узора и улучшение характеристик работы при выделении цепей. Время работы алгоритма также планируется сократить. Возможно рассмотрение приложений построения скелета серого изображения относительно других прикладных задач, например, в карто\-графии.

\begin{thebibliography}{1}
\bibitem{bib1}
    \BibAuthor{Местецкий\;Л.\,М.}
    Векторизация бинарных растровых изображений на основе аппроксимации~//
    Доклады Х Всероссийской конференции <<Математические методы распознавания>> (ММРО-10). Москва, 2001.
\bibitem{bib2}
    \BibAuthor{Местецкий\;Л.\,М.}
    Непрерывный скелет бинарного растрового изображения~//
    Труды международной конференции <<Графикон-98>>. Москва, 1998.
\bibitem{bib3}
    \BibAuthor{Котик\;С.\,В., Местецкий\;Л.\,М.}
    Сжатие  полутоновых изображений рукописных документов на основе кодирования по изолиниям яркости~//
    Доклады XII Всероссийской конференции <<Математические методы распознавания>> (ММРО-12). Москва, 2005
\bibitem{bib4}
    \BibAuthor{Mersal\;S.\,S., Darwish\;A.\,M.}
    A New Parallel Thinning Algorithm For Gray Scale Images~//
    Works of  IEEE-EURASIP Workshop on Nonlinear Signal and Image Processing "--- NSIP'99. Antalya, 1999.
\bibitem{bib5}
    \BibAuthor{Котик\;С.\,В.}
    Скелетизация полутонового изображения на примере обработки изображений отпечатков пальцев~//
    Таврический вестник информатики и математики. Симферополь, 2008.
\end{thebibliography}

\ACCEPTNOTE
\end{document}
