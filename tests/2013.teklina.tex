\documentclass[twoside]{article}
\usepackage{mmro16}

\begin{document}
\title
    {Синтез простейших нелинейных систем квазиинвариантного управления с заданными свойствами методами распознавания образов}
\author
    {Теклина~Л.\,Г., Котельников~И.\,В., Гельфер~И.\,С.} % список авторов (с инициалами) для оглавления
    [Теклина Лариса Григорьевна \coauthor Котельников Игорь Вячеславович \coauthor Гельфер Ирина Самуиловна] % список авторов (Имя Отчество полностью) для заголовка тезисов
\email
    {neymark@pmk.unn.ru}
\organization
    {Нижний Новгород, Нижегородский государственный университет им.~Н.~И.~Лобачевского}
\maketitle

Синтез систем  квазиинвариантного управления, систем, которые должны не устранять возникающие ошибки, а предотвращать их, т.\,е. сделать объект управления невосприимчивым (инвариантным) к внешним воздействиям~--- весьма актуальная проблема. Квазиинвариантность означает малость ошибки управления в установившемся режиме. Для решения этой проблемы используется новый подход, основанный на постановке и решении задачи синтеза методами распознавания образов с активным экспериментом. Отличительной особенностью нового подхода является переход от классических методов исследования к новой методике, в которой сложные расчеты заменены достоверным математическим экспериментом с последующей обработкой результатов экспериментов методами интеллектуального анализа данных, работающими в пространствах большой размерности. На базе этого подхода разработана новая методика синтеза \emph {линейных} систем квазиинвариантного управления. Применение методов распознавания образов позволило синтезировать системы управления с заданными свойствами, характеризующими не только установившийся режим, но и качество переходного процесса.

Однако, и теоретически, и экспериментально показано, что в рамках линейных систем удовлетворить всем требованиям, выдвигаемым к системе управления, часто бывает сложно, но возможно при выборе нелинейной стратегии управления. Работа посвящена дальнейшему развитию нового подхода к синтезу систем квазиинвариантного управления путем расширения методики синтеза линейных систем на область \emph {нелинейных} систем управления, что позволяет преодолеть главный недостаток линейных систем~--- большие значения функции управления в переходном процессе.

Работа поддержана грантом РФФИ \No\,11-01-00379.

\begin{thebibliography}{1}
\bibitem{1}
    \emph{Теклина\;Л.\,Г., Котельников\;И.\,В., Гельфер\;И.\,С.}
    Применение методов распознавания образов для синтеза кусочно-линейных систем квазиинвариантного управления~//
    Машинное обучение и анализ данных,
    %М.:~Изд-во ВЦ РАН им.~Дородницына, 
    2013.
    \url{http://jmlda.org/papers/index.php/JMLDA/authors/submission/51}.
\end{thebibliography}


\end{document}
