\documentclass[twoside]{article}
\usepackage{mmro}

\begin{document}
\title[Архитектура и разработка системы имитационного моделирования]%
    {Архитектура и разработка системы имитационного моделирования
    для многофакторных моделей социальной динамики}
\author{Бабкин~Э.\,А., Козырев~О.\,Р.}
\email{okozyrev@hse.nnov.ru}
\organization{Нижний Новгород, Государственный университет, Высшая школа экономики}
\maketitle

\section{Постановка задачи}
Мы сосредоточим наши исследования на решении важной научной проблемы по
разработке единой методологии создания архитектуры информационно"=вычислительных
средств для математического, информационного и компьютерного моделирования
сложных процессов в социальной динамике с учетом влияния многих факторов
внешней среды.

Для эффективного разрешения указанных проблем требуются
информационно"=вычислительные средства нового поколения на основе современных
технологий компонентных и распределенных вычислений. При разработке систем
имитационного моделирования, за исключением единичных примеров системного,
концептуального подхода к построению моделей, часто выбирается традиционный
путь. Знания о структуре математической модели и границах ее применимости
выражаются в виде описаний на естественном языке, а также неявно содержатся
в~программном коде, который самостоятельно создается авторами моделей.
Это приводит к ряду серьезных проблем (практически невозможно получить
отчуждаемое описание семантики соответствующей математической модели;
ручное кодирование моделей противоречит современным тенденциям развития
программной инженерии).

Особые требования предъявляются к моделям социальной
и культурной сфер. Здесь для эффективного разрешения указанных проблем
требуются разработанные модели и информационно"=вычислительные средства нового
поколения на основе знаний~\cite{2,3}. Они должны позволять исследователям
проводить разработку и интеграцию математических моделей, независящих от
особенностей программной реализации, с последующей автоматизированной
генерацией кода для проведения численных экспериментов в различных системах
имитационного моделирования. Необходимо наличие единого системного подхода ко
всем этапам математического моделирования, что обуславливает большую актуальность
разработки методологии унифицированного семантического описания содержательных
моделей и алгоритмов по их преобразованию в~различные формы.

Отдельные задачи
абстрактного системного описания математических моделей социально"=экономических
систем решаются в рамках подхода «Системный анализ развивающейся экономики»,
развиваемого ВЦ~РАН. Однако этот подход нацелен на описание ограниченного числа
типов моделей экономики, а~решение по сопровождению жизненного цикла распределенных
систем моделирования и автоматизированная генерация программного кода агентов
для многоагентных имитационных систем там отсутствуют.

Определенной степенью
близости обладает также проект AuctionBot Мичиганского университета, США, но
в нем отсутствуют общие принципы описания моделей взаимодействия.
В~ряде международных проектов ведется работа по систематизации
и~предоставлению доступа к~различным информационным ресурсам
на~основе протокола Z39.50.
Наиболее известными из них являются проекты Aquarelle и~AHDS.

\section{Предполагаемые подходы}

В изучении феноменов социальной динамики
важную роль играют функционально"=структурные модели.
При этом одним из наиболее перспективных направлений считается
многоаспектное математико"=информационное моделирование,
воплощенное в форме распределенных имитационных систем
на основе парадигмы взаимодействия индивидуальных сущностей (individual-based systems).
Для построения и анализа таких моделей
с помощью методов имитационного моделирования
требуется наличие единой информационно"=вычислительной системы,
предоставляющей удаленным пользователям удобный интерфейс
для самостоятельного расширения возможностей взаимодействия с системой,
обладающей развитыми средствами автоматизированной разработки и анализа
имитационных многоаспектных моделей.
Такой подход носит название генеративного описания моделей
(generative models description).

Задача, которая решается в данном проекте,
состоит в построении методологии моделирования и программной архитектуры
распределенной информационно"=вычислительной системы
для генеративного описания и изучения
многофакторных моделей социальной динамики.
Эти функции предполагается эффективно реализовать
на основе развития принципов многоагентного моделирования
и~средств распределенного программирования в~стандарте CORBA,
развивая полученные ранее результаты в~области
распределенных систем имитационного моделирования~\cite{2,3}.

Результатом явится достижение следующих научных целей:
\begin{itemize}
\item
    создание новой методологии построения распределенных
    систем имитационного моделирования на основе совместного применения
    CORBA и многоагентных систем;
\item
    получение объективных критериев производительности
    класса подобных систем имитационного моделирования;
\item
    разработка новых математических моделей и алгоритмов
    масштабируемых вычислительных экспериментов
    на моделях социальной динамики в локальных сетях.
\end{itemize}

Новизна решаемой задачи обусловлена
отсутствием соответствующего программного и информационного обеспечения
и цельной методологии проведения имитационных экспериментов
в~социально"=экономических исследованиях.
Нами развиваются методы специализации группы современных информационных технологий
(CORBA и многоагентные системы) для решения специфических задач,
возникающих в ходе проведения имитационных экспериментов
с многофакторными функционально"=структурными моделями социальной динамики.
Особенность используемых подходов заключается в применении теоретических принципов
декларативного унифицированного описания семантики математических моделей,
а~также алгоритмов интеграции и трансформации декларативных
описаний моделей в~набор программных агентов
для «сквозной» автоматизации задач имитационного моделирования.

Важным научным вкладом предлагаемого проекта
будут являться новые принципы описания структуры и~поведения
распределенной информационно"=вычислительной системы,
использующей архитектуру CORBA и~технологию многоагентных вычислений,
с~учетом требований математико"=информационного моделирования социальных систем.
Эта система предположительно будет состоять из трех составных частей:
\begin{itemize}
\item
    подсистема декларативных описаний семантики математических моделей;
\item
    подсистема сопровождения жизненного цикла;
\item
    подсистема проведения имитационных экспериментов и справочная подсистема (Web-портал).
\end{itemize}

Для взаимодействия подсистем планируется использовать
объектно"=ориентированные распределенные технологии:
связь с имитационным сервером реализовать с~помощью технологии CORBA (TAO ORB),
Web"=портал является сервером приложений,
построенным на принципах трехуровневой архитектуры
и~обеспечивающим четыре вида открытых интерфейсов
для внешних пользователей и внутренних подсистем
(динамический HTML, Java-апплеты, EJB, SOAP-CORBA).
Банк знаний обеспечивает функции специализированного графического редактора моделей
и~функции по долговременному хранению описаний моделей в~СУБД.
Для реализации банка знаний используется язык Java и~объектно"=реляционная СУБД PostgreSQL.
В~качестве лингвистических средств в~банке знаний используются
известные стандарты представления знаний: XML, RDF, DAML+OIL.
Будет также обеспечена возможность экспорта знаний в~различном представлении
(онтологии, модели) из ряда широко распространенных инструментальных средств
(в~частности, из системы Protege).
Для построения эффективного имитационного
сервера применяется технология многоагентных распределенных систем
с~программной реализацией на основе расширения системы SWARM.

Таким образом, автоматизированная генерация кода по моделям
будет приводить к~созданию наборов автономных агентов.
Их~реализация будет сохраняться в~информационно"=логической системе
для повторного использования в составе других моделей.
Работа выполнена при финансовой поддержке гранта РФФИ №\,07-07-00058.

\begin{thebibliography}{1}
\bibitem{1}
    \BibAuthor{Бабкин~Э.\,А., Козырев~О.\,Р.}
    Система моделирования микроэкономических сценариев:
    общая концепция и принципы программной реализации //
    Известия АИН РФ. Сер. Прикладная математика и информатика, Т. 6.
    Москва, Н.~Новгород, 2006. "--- С.~22--30.
\bibitem{2}
    \BibAuthor{Бабкин~Э.\,А., Козырев~О.\,Р., Куркина~И.\,В.}
    Принципы и алгоритмы искусственного интеллекта //
    Н.~Новгород: Изд-во НГТУ, 2006. "--- 132~С.
\bibitem{3}
    \BibAuthor{Бабкин~Э.\,А., Козырев~О.\,Р.}
     Методы представления знаний и алгоритмы поиска
     в~задачах искусственного интеллекта. "---
     Н.~Новгород, Изд-во Талам, 2005. "--- 146 С.
\bibitem{4}
    \BibAuthor{Babkin~E.\,A., Kozyrev~O.\,R., Logvinova~K.\,V., Zubov~M.\,L.}
     Ontology"=based Modeling of Micro Economics Scenarios //
     Proceedings of BIR-2004, Shaker Verlag, 2004. "--- p.~33--44.
\end{thebibliography}

\end{document}
