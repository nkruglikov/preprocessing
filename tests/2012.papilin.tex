\documentclass[twoside]{article}
\usepackage{iip9}

\begin{document}
\title
    {Кооперативные стратегии для возможностных моделей биматричных игр}
\author
    {Папилин~С.\,С., Пытьев~Ю.\,П.}
\email
    {papilin@physics.msu.ru, yuri.pytyev@gmail.com}
\organization
    {Москва, Московский государственный университет имени М.\,В.\,Ломоносова}
\abstract{
    В докладе исследуются кооперативные стратегии в~возможностных моделях
    биматричной игры двух субъектов~A и~B.
Исследуется существование ситуаций, реализующих наиболее
благоприятный для обоих игроков результат, и~ситуаций, оптимальных
по Парето. Находятся условия на матрицы, при которых игрокам может
быть выгодна кооперация.
    }
\thanks{Работа выполнена при финансовой поддержке РФФИ, проект \No\,11-07-00722-а.}
\titleEng{
    Cooperative strategies for possibilistic models of bimatrix games}
\authorEng{Papilin~S.\,S., Pyt'ev~Yu.\,P.}
\organizationEng{
    M.\,V.\,Lomonosov Moscow State University, Moscow, Russia}
\abstractEng{
    This paper investigates the cooperative
strategy in the possibilistic models of bimatrix games of two
entities A~and~B. We investigate the existence of situations which
implement the most favorable outcome for both players, and of Pareto
efficient situations. We find the set of matrices which let the
players have advantageous cooperation.}

\maketitle

В докладе рассмотрена матричная игра, в~которой решения игроков
влияют на возможность некоторых событий~$V$ и~$W$, которые игроки~A
и B~соответственно считают своими выигрышами или проигрышами
(в~зависимости от постановки задачи).

В работе~\cite{article} анализ такой игры в~некооперативном варианте был проведен с позиций первого
варианта теории возможностей. Согласно доказанной в~\cite{article} теореме, среди оптимальных
стратегий может и~не быть четких, поэтому в~работе~\cite{article} для реализации нечетких решений
игроков была предложена статистическая рандомизация их стратегий, не свойственная возможностной
теории оптимальных решений~\cite{book} и~наделяющая ее недостатками вероятностной модели, например,
необходимостью многократного повторения испытаний (в данном случае "--- актов игры) для реализации
модели. В той же статье было отмечено, что из-за того, что независимость вероятностная не
равносильна независимости возможностной, для практической реализации нечетких стратегий игроков
требуется <<крупье>>, получающий указания от игроков и~сообщающий им их случайные решения.

Поэтому представляется естественным рассмотреть кооперативные игры, что и~делается в~настоящей
статье с~привлечением также второго варианта теории возможностей~\cite{book}. Кроме того, что
кооперация может увеличить возможности выигрышей игроков, она может позволить получить четкие
оптимальные стратегии, что снимет вопрос о~рандомизации. Впрочем, как показано в~этой статье,
полностью удовлетворяющая обоих игроков четкая кооперативная стратегия существует не в~любой игре.
    %
\section{Возможностная модель\\ биматричной игры}
Кооперативная стратегия игроков подразумевает, что они договариваются, с~какими возможностями
$\mathrm{p}^\mathrm{AB}_{ij}$ им выбирать строку~$i$ и~столбец~$j$ в~матрице игры. В~\cite{article}
игроки использовали некооперативные стратегии $\mathrm p^\mathrm A_i$ и $\mathrm p^\mathrm B_j$,
и~строка~$i$ и~столбец~$j$ выбирались с возможностью $\mathrm p^\mathrm{A}_i{\times}\mathrm
p^\mathrm B_j$, $i=1,\ldots,m$,; $j=1,\ldots,n$.
    %
\begin{Definition}
В модели нечеткого акта игры игроки~A и~B представлены каноническим для прост\-ранс\-тва
с~возможностью $(\Omega_m{\times}\Omega_n,\mathcal{P}(\Omega_m{\times}\Omega_n)$,
$\mathrm{P}^{\mathrm{AB}})$ нечетким элементом $\delta(\alpha,\beta)${\rm,} определяющим
возможность $\mathrm{P}^{\mathrm{AB}}(\alpha=i,\ \beta=j)=\mathrm{p}^{\mathrm{AB}}_{ij}$ принятия
$i$-го решения игроком~A и~$j$-го решения игроком B{\rm,} $i=1,\ldots,m$,\; $j=1,\ldots,n$,
в~рассматриваемом контексте называемую нечеткой кооперативной стратегией игроков~A и~B{\rm,}
$\mathcal{P}^{\mathrm{AB}}=\{\mathrm{p}^{\mathrm{AB}}=(\mathrm{p}^{\mathrm{AB}}_{11},\ldots,
\mathrm{p}^{\mathrm{AB}}_{mn}),\ 0\le\mathrm{p}^{\mathrm{AB}}_{ij}\le1,\ i=1,\ldots,m,\
j=1,\ldots,n,\  \max_{ij}\mathrm{p}^{\mathrm{AB}}_{ij}=1\}$ "--- множество всех нечетких
кооперативных стратегий.

Каждая кооперативная стратегия удовлетворяет условию нормировки
    %
    \begin{equation}\label{norm}
\max_{i,j}\mathrm{p}^{\mathrm{AB}}_{ij}=1.
    \end{equation}

Результатом фазифицированного акта игры назовем возможности
    \[
\mathrm{P}(V,\mathrm{p}^\mathrm{AB})=\max_{i,j}\mathrm{P}_\mathrm{A}^{(i,j)}(V)\times\mathrm{p}^\mathrm{AB}_{ij}
    \]
<<выигрыша>> (<<проигрыша>>) V игрока A и
    \[
\mathrm{P}(W,\mathrm{p}^\mathrm{AB})=\max_{i,j}\mathrm{P}_\mathrm{B}^{(i,j)}(W)\times\mathrm{p}^\mathrm{AB}_{ij}
    \]
<<выигрыша>> (<<проигрыша>>) W игрока~B{\rm,} если игроки используют
стратегию $\mathrm{p}^\mathrm{AB}\in\mathcal{P}^\mathrm{AB}$.
\end{Definition}

Здесь и далее символ <<$\times$>> означает минимум в~первом варианте
теории возможностей и <<обычное умножение>> во втором.

Пусть заданы две матрицы переходных возможностей, матричные элементы которых
    \[
    \begin{aligned}
a_{ij}&\triangleq\mathrm{P}_\mathrm{A}^{(i,j)}(V)=\mathrm{P}(V\cond\alpha=i,\
\beta=j);
    \\
b_{ij}&\triangleq\mathrm{P}_\mathrm{B}^{(i,j)}(W)=\mathrm{P}(W\cond\alpha=i,\
\beta=j).
    \end{aligned}
        \]
В частном случае противоположных событий ($V=\Omega\setminus W$) на
матрицы налагается дополнительное условие:
$\max(a_{ij},b_{ij})=1$,\; $i,j\in N$. Соответственно
    %
    \begin{align}
\mathrm{P}(V\cond\mathrm{p}^\mathrm{AB})&=\max_{i,j}a_{ij}\times\mathrm{p}^\mathrm{AB}_{ij}\triangleq
A(\mathrm{p}^\mathrm{AB})\,;\label{PV}
    \\
\mathrm{P}(W\cond\mathrm{p}^\mathrm{AB})&=\max_{i,j}b_{ij}\times\mathrm{p}^\mathrm{AB}_{ij}\triangleq
B(\mathrm{p}^\mathrm{AB})\label{PW}
    \end{align}
    %
есть результат акта игры как функция нечеткой кооперативной стратегии игроков.

\section{Непротивоположные события}
Для рассматриваемой модели игры можно пос\-та\-вить две задачи: максимизации и~минимизации. В
задаче максимизации цель игрока~A "--- максимизировать $\mathrm{P}(V\cond\mathrm{p}^\mathrm{AB})$,
а~цель игрока~B "--- максимизировать $\mathrm{P}(W\cond\mathrm{p}^\mathrm{AB})$. В~задаче
минимизации цель игрока~A "--- минимизировать $\mathrm{P}(W\cond\mathrm{p}^\mathrm{AB})$, а~цель
игрока~B "--- минимизировать $\mathrm{P}(V\cond\mathrm{p}^\mathrm{AB})$.
    %
\paragraph{Задача максимизации.}
Функции $\mathrm{P}(V\cond\mathrm{p}^\mathrm{AB})$ и~$\mathrm{P}(W\cond\mathrm{p}^\mathrm{AB})$ не
убывают по каждому
    \[
\mathrm{p}^\mathrm{AB}_{ij},\quad 1\le i\le m,\quad 1\le j\le n.
    \]
Поэтому в задаче максимизации максимальные, оптимальные для игроков возможности обоих событий
реализует тривиальная стратегия из всех единиц. Эти оптимальные возможности:
    \[
\mathrm{P}_{\max}(V)\stackrel{\mathrm{def}}{=}\bar
a=\max_{i,j}a_{ij}, \quad
\mathrm{P}_{\max}(W)\stackrel{\mathrm{def}}{=}\bar
b=\max_{i,j}b_{ij}.
    \]

Оптимальной (для обоих игроков) будем называть любую стратегию, реализующую оптимальные возможности
для обоих игроков одновременно. Учитывая~\eqref{PV} и~\eqref{PW}, получаем условие на оптимальность
стратегии:
    \begin{equation}\label{optmax}
\max_{i,j}a_{ij}\times\mathrm{p}^\mathrm{AB}_{ij}=\bar a\,;
    \quad
\max_{i,j}b_{ij}\times\mathrm{p}^\mathrm{AB}_{ij}=\bar b\,.
    \end{equation}

Определим множества пар $(i,j)$ (со\-от\-ветс\-тву\-ющих, кстати,
четким стратегиям)
    \begin{equation}\label{opi}
    \hspace*{-7pt}
    \begin{aligned}
&\omega\stackrel{\mathrm{def}}{=}\left\{(i,j)\in(1,\dots,m){\times}(1,\dots,n)%|
%1\leqslant i\leqslant m,1\leqslant j\leqslant n
\right\};
    \\
&\bar\alpha\stackrel{\mathrm{def}}{=}\left\{(i,j)\in(1,\dots,m){\times}(1,\dots,n)\cond
a_{ij}=\bar a\right\};
    \\
&\bar\beta\stackrel{\mathrm{def}}{=}\left\{(i,j)\in(1,\dots,m){\times}(1,\dots,n)\cond
b_{ij}=\bar b\right\};
    \\
&\bar\gamma\stackrel{\mathrm{def}}{=}\bar\alpha\cap\bar\beta={}
    \\
&\!=\!\left\{(i,j)\!\in\!(1,\dots,m){\times}(1,\dots,n)\cond
a_{ij}\!=\!\bar a,\, b_{ij}\!=\!\bar b\right\}.
    \end{aligned}
    \end{equation}
$\bar\alpha$ и $\bar\beta$, очевидно, непусты (хотя бы одно число
в~конечной матрице равно максимуму по этой \mbox{матрице}).

Поскольку в обоих вариантах теории воз\-мож\-ностей
$x{\times}\mathrm{p}\le x$, то в~условии~\eqref{optmax} достаточно
вычислять максимум лишь по точкам из множеств~$\bar\alpha$
и~$\bar\beta$ и~требовать достижения максимума хотя бы в~одной.
Условие на оптимальность переписывается в~виде
    \begin{equation}\label{expoptmax}
    \begin{aligned}
\exists(i,j)&\in\bar\alpha\colon\bar
a{\times}\mathrm{p}^\mathrm{AB}_{ij}=\bar a\,;
    \\
\exists(i,j)&\in\bar\beta\colon\bar
b{\times}\mathrm{p}^\mathrm{AB}_{ij}=\bar b\,.
    \end{aligned}
    \end{equation}
    %
$\bar\beta$ является множеством всех оптимальных для обоих игроков четких стратегий, так как каждая
стратегия из~$\bar\gamma$ реализует две оптимальные воз\-мож\-ности и~ни одна стратегия не
из~$\bar\gamma$ не реализует их одновременно. При этом все четкие стратегии в~$\bar\gamma$
равноценны и никакой нельзя отдать пред\-поч\-те\-ние.

Условимся использовать следующие обозначения:
$(i_1,j_1)\gg(i_2,j_2)\Leftrightarrow(i_2,j_2)\ll(i_1,j_1)$, если $a_{i_1j_1}\ge a_{i_2j_2}$,\;
$b_{i_1j_1}\ge b_{i_2j_2}$ и~либо $a_{i_1j_1}>a_{i_2j_2}$, либо $b_{i_1j_1}>b_{i_2j_2}$. Определим
множество пар (четких стратегий), оптимальных по Парето:
    %
    \begin{multline}
\bar
o\stackrel{\mathrm{def}}{=}\big\{(i,j)\in(1,\dots,m){\times}(1,\dots,n)\cond\nexists(i',j')\in{}
    \\
{}\in{}(1,\dots,m){\times}(1,\dots,n)\colon(i',j')\gg(i,j)\big\}.\label{oo}
    \end{multline}
    %
Поскольку матрицы имеют конечные размеры, то~$\bar o$ всегда не пусто. Назовем это множество
\emph{ядром игры в задаче максимизации}. Очевидно, что если~$\bar\gamma$ не\-пус\-то, то оно
совпадает с~$\bar o$ (т.\,к.~$\bar\gamma$ "--- это множество, соответствующее максимумам по обеим
матрицам). Если~$\bar\gamma$ пусто, то в~$\bar o$ как минимум два элемента (один из~$\bar\alpha$,
другой из~$\bar\beta$ "--- множеств, соответствующих максимумам по матрицам~$a_{ij}$ и~$b_{ij}$
со\-от\-ветс\-твен\-но), и совместную четкую стратегию игрокам остается выбирать из $\bar o$.
    %
\paragraph{Задача минимизации.}
В задаче минимизации игроки стремятся уменьшить до нуля как можно больше возможностей
$\mathrm{p}^\mathrm{AB}_{ij}$, но хотя бы одна из них должна оставаться равной единице для
выполнения нормировки~\eqref{norm}. Поэтому среди оптимальных для отдельных игроков стратегий есть
четкие, а~оптимальные для игроков~A и~B соответственно возможности равны:
    %
    \[
\mathrm{P}_{\min}(V)\stackrel{\mathrm{def}}{=}\underline
a=\min_{i,j}a_{ij}\,;
    \quad
\mathrm{P}_{\min}(W)\stackrel{\mathrm{def}}{=}\underline
b=\min_{i,j}b_{ij}\,.
    \]

Оптимальной (для обоих игроков) будем называть любую стратегию, реализующую оптимальные возможности
для обоих игроков одновременно. Учитывая~\eqref{PV}, \eqref{PW} получаем условие на оптимальность
стратегии:
    \begin{equation}\label{optmin}
\max_{i,j}a_{ij}{\times}\mathrm{p}^\mathrm{AB}_{ij}=\underline a\,;
    \quad
\max_{i,j}b_{ij}{\times}\mathrm{p}^\mathrm{AB}_{ij}=\underline b\,.
    \end{equation}

Определим множества пар $(i,j)$
    %
    \begin{equation}\label{upi}
\hspace*{-7pt}
    \begin{aligned}
&\underline\alpha\stackrel{\mathrm{def}}{=}\left\{(i,j)\in(1,\dots,m){\times}(1,\dots,n)\cond
a_{ij}=\underline a\right\};
    \\
&\underline\beta\stackrel{\mathrm{def}}{=}\left\{(i,j)\in(1,\dots,m){\times}(1,\dots,n)\cond
b_{ij}=\underline b\right\};
    \\
&\underline\gamma\stackrel{\mathrm{def}}{=}\underline\alpha\cap\underline\beta={}
    \\
&=\left\{(i,j)\!\in\!(1,\dots,m){\times}(1,\dots,n)\cond
a_{ij}\!=\!\underline a,\, b_{ij}\!=\!\underline b\right\}.
    \end{aligned}
    \end{equation}
$\underline\alpha$ и~$\underline\beta$, очевидно, не пусты (хотя бы одно число в~конечной матрице
равно минимуму по этой матрице).

Поскольку в~обоих вариантах теории воз\-мож\-ностей $x{\times}\mathrm{p}\le x$, то
в~условии~\eqref{optmin} точки из множеств $\underline\alpha$ и~$\underline\beta$ можно не
учитывать. Условие на оптимальность переписывается в виде:
    %
    \begin{equation}\label{expoptmin}
    \begin{aligned}
\forall(i,j)&\in\neg\underline\alpha\colon
a_{ij}{\times}\mathrm{p}^\mathrm{AB}_{ij}\le\underline a\,;
    \\
\forall(i,j)&\in\neg\underline\beta\colon
b_{ij}{\times}\mathrm{p}^\mathrm{AB}_{ij}\le\underline b\,,
    \end{aligned}
    \end{equation}
где $\neg\underline\alpha=\omega\setminus\underline\alpha$,
$\neg\underline\beta=\omega\setminus\underline\beta$.
    %
$\underline\gamma$ является множеством всех оптимальных для обоих
игроков четких стратегий, т.\,к. каждая стратегия
из~$\underline\gamma$ реализует две оптимальные возможности и~ни
одна стратегия не из~$\underline\gamma$ не реализует их
одновременно. При этом все четкие стратегии в~$\underline\gamma$
равноценны и никакой нельзя отдать предпочтение.

Также определим множество пар (четких стратегий), оптимальных по Парето:
    \begin{multline}\label{uo}
\underline
o\stackrel{\mathrm{def}}{=}\big\{(i,j)\in(1,\dots,m){\times}(1,\dots,n)\cond\nexists(i',j')\in
    \\
\in(1,\dots,m){\times}(1,\dots,n)\colon (i',j')\ll(i,j)\big\}.
    \end{multline}
Поскольку матрицы имеют конечные размеры, то~$\underline o$ всегда не пусто. Назовем его
\emph{ядром игры в задаче минимизации}. Очевидно, если $\underline\gamma$ не пусто, то оно
совпадает с~$\underline o$ (т.\,к.~$\underline\gamma$ "--- это множество, соответствующее минимумам
по обеим матрицам). Если~$\underline\gamma$ пус\-то, то в~$\underline o$ как минимум два элемента
(один из~$\underline\alpha$, другой из~$\underline\beta$ "--- множеств, соответствующих минимумам
по матрицам~$a_{ij}$ и~$b_{ij}$ соответственно), и~совместную четкую стратегию игрокам остается
выбирать из~$\underline o$.

\section{Противоположные события}
    %
В этом случае $W=\Omega\setminus V$. На матрицы~$\mathrm A$ и~$\mathrm B$ налагается дополнительное
условие:
    \[
\max(a_{ij},b_{ij})=1,\quad i=1,\ldots,m,\quad j=1,\ldots,n.
    \]
        %
\paragraph{Задача максимизации.}
Хотя бы в одной матрице есть единицы, поэтому $\max(\bar a,\bar
b)=1$.

В условии на множество всех оптимальных стратегий~\eqref{expoptmax} следует учесть, что
$1{\times}\mathrm{p}^\mathrm{AB}_{ij}=1\Leftrightarrow\mathrm{p}^\mathrm{AB}_{ij}=1$,\;
$i=1,\ldots,m$,\; $j=1,\ldots,n$ как в~первом, так и~во втором варианте теории возможностей.

Если $\bar a<1$, то все числа в~матрице $a_{ij}$ меньше единицы, следовательно, все числа
в~матрице~$b_{ij}$ равны единице. Матрицу, состоящую исключительно из единиц, будем называть
безразличной. В~этом случае $\bar\gamma=\bar\alpha$, и,~следовательно, не пусто. В~задаче есть
решение в~виде оптимальной четкой стратегии. Если же $\bar b<1$, то безразличной является матрица
$a_{ij}$, и аналогично $\bar\gamma=\bar\beta$ не пусто, и~в~задаче есть решение в~виде оптимальной
четкой стратегии. Можно сказать, что игрок с~безразличной матрицей не может изменить собственную
прибыль, но может помочь другому, выбрав наиболее выгодный для того вариант. Эти случаи тривиальны.

В нетривиальном случае $\bar a=\bar b=1$ имеем
$\bar\gamma=\{(i,j)\in(1,\dots,m){\times}(1,\dots,n)\cond a_{ij}=1,\
b_{ij}=1\}$, и~мно\-жест\-во $\bar\gamma$ может быть пустым.
    %
\paragraph{Задача минимизации.}
Если $\underline a=1$, то вся матрица $a_{ij}$ заполнена единицами и~является безразличной. Тогда
$\underline\gamma=\underline\beta$ не пусто, и~в~задаче есть решение в~виде оптимальной четкой
стратегии. Аналогично, если $\underline b=1$, то вся матрица $b_{ij}$ заполнена единицами,
$\underline\gamma=\underline\alpha$, и~в~задаче есть решение в~виде оптимальной четкой стратегии.
Эти случаи тривиальны.

В нетривиальном случае $\underline a<1$,\; $\underline b<1$ мно\-жест\-ва~$\underline\alpha$
и~$\underline\beta$ не пересекаются (на местах, где в~одной матрице стоят числа, меньшие единицы,
в~другой стоят единицы), то есть $\neg\underline\alpha\cup\neg\underline\beta=\omega$. Согласно
нормировке~\eqref{norm}, хотя бы в~одной из точек~$\omega$ возможность равна единице, и~эта точка
попадает либо в~$\neg\underline\alpha$, либо в~$\neg\underline\beta$. Поскольку в~обоих вариантах
теории возможностей $x{\times}1=x$, то нарушается хотя бы одно условие в~\eqref{expoptmin}. Это
означает, что оптимальных для обоих игроков стратегий в~такой игре нет.

Как уже сказано выше, в~тривиальных случаях $\underline a=1$ или $\underline b=1$ четкая
оптимальная стратегия есть.

В~нетривиальном случае $\underline a<1$ и~$\underline b<1$ оптимальных стратегий нет, в~том числе
четких. Так как~$\underline\alpha$ и~$\underline\beta$ не пересекаются, то~$\underline\gamma$
пусто. Мно\-жест\-во всех точек~$\omega$ можно разбить на три не\-пе\-ре\-се\-кающиеся части:
в~первой $a_{ij}=1$,\; $b_{ij}=1$ (эта часть может быть пустой), во второй $a_{ij}<1$,\;
$b_{ij}=1$, в~третьей $a_{ij}=1$,\; $b_{ij}<1$. Отсюда очевидно, что $\underline o$ является
объединением~$\underline\alpha$ и~$\underline\beta$, подмножеств второй и~третьей частей
соответственно:
    \[
    \begin{aligned}
\underline o&=\underline\alpha\cup\underline\beta\,;
    \\
\underline\alpha&=\left\{(i,j)\cond a_{ij}=\underline a,\ b_{ij}=1\right\};
    \\
\underline\beta&=\left\{(i,j)\cond a_{ij}=1,\ b_{ij}=\underline
b\right\}.
    \end{aligned}
    \]

Выбор четкой стратегии из~$\underline\alpha$ реализует наилучший вариант для игрока~B и~наихудший
вариант для игрока~A; выбор стратегии из~$\underline\beta$ реализует наилучший вариант для игрока~A
и~наихудший вариант для игрока~B. Таким образом, после ограничения выбора множеством $\underline o$
дальнейшая кооперация становится невозможной.

\section{Теоремы о существовании четких\\ оптимальных стратегий}
    %
Если в задаче максимизации или минимизации существует четкая совместная оптимальная стратегия
($\bar\gamma$ или~$\underline\gamma$ соответственно не пусты), то она является точкой равновесия
(в~ней обе матрицы имеют экстремум, и~такой же экстремум в~ней имеют проходящие через эту точку
строка и~столбец в~каждой матрице). Обобщим результаты в~теоремах о~выборе оптимальных четких
стратегий.
   %
\paragraph{Задача максимизации.}
\begin{Theorem}
Множество $\bar\gamma${\rm,} определенное в~\eqref{opi}{\rm,} является подмножеством множества всех
точек равновесия из четких стратегий в~задаче максимизации. Все четкие стратегии, входящие
в~$\bar\gamma${\rm,} дают одинаковый{\rm,} наилучший для обоих игроков{\rm,} результат (таким
образом{\rm,} при не пустом~$\bar\gamma$ конфликт интересов в~игре исчезает), и~кооперация (под
кооперацией понимается такое сотрудничество игроков, при котором каждый из них не обязательно
стремится к~наилучшему лично для него результату) необязательна для достижения наилучшего
результата. Однако для уверенности в~реализации наилучшего варианта игрокам может понадобиться
обменяться информацией о своих стратегиях.

В~задаче с~противоположными событиями $\bar\gamma$ не пусто лишь в~тривиальных случаях
с~безразличной матрицей или в~нетривиальном случае при наличии точек $(i,j)\cond
a_{ij}=1,b_{ij}=1$.

Если множество~$\bar\gamma$ пусто{\rm,} то возможна ситуация{\rm,} когда ни одна из точек
равновесия из четких стратегий не входит в~$\bar o${\rm,} определенное в~\eqref{oo}. Если
считать{\rm,} что при отсутствии кооперации игроки придерживаются точек равновесия из четких
стратегий{\rm,} то кооперация позволит им выбрать вместо точки равновесия точку из~$\bar o${\rm,}
тем самым улучшив по сравнению с~точкой равновесия результат как минимум одного игрока. Следует
заметить{\rm,} что{\rm,} т.\,к. при пустом~$\bar\gamma$ в~$\bar o$ как минимум два элемента{\rm,}
игрокам предстоит договориться о~выборе конкретного элемента из~$\bar o$. В этом случае цель
кооперации достигнута.

%\balance

Если же при пустом~$\bar\gamma$ среди точек равновесия из четких стратегий есть входящие в~$\bar
o${\rm,} то при отсутствии кооперации{\rm,} но присутствии обмена информацией игроки договорятся
о~выборе одной из них{\rm,} предпочитая их точкам равновесия{\rm,} не входящим в~ядро. Кооперация
позволяет сместить выбор на точки~$\bar o${\rm,} не являющиеся точками равновесия{\rm,} но выгодно
ли это будет игрокам{\rm,} сказать нельзя.
\end{Theorem}

\paragraph{Задача минимизации.}
    \begin{Theorem}
Множество $\underline\gamma${\rm,} определенное в~\eqref{upi}{\rm,} является подмножеством
множества всех точек равновесия из четких стратегий в~задаче минимизации. Все четкие
стратегии{\rm,} входящие в~$\underline\gamma${\rm,} дают одинаковый{\rm,} наилучший для обоих
игроков{\rm,} результат (таким образом{\rm,} при не пустом~$\underline\gamma$ конфликт интересов в
игре исчезает){\rm,} и~кооперация (под кооперацией понимается такое сотрудничество игроков{\rm,}
при котором каждый из них не обязательно стремится к~наилучшему лично для него результату)
необязательна для достижения наилучшего результата. Однако для уверенности в~реализации наилучшего
варианта игрокам может понадобиться обменяться информацией о~своих стратегиях.

В~задаче с~противоположными событиями~$\underline\gamma$ не пусто лишь в~тривиальных случаях
с~безразличной матрицей. В~таком случае обмен информацией между игроками позволит им достичь
оптимального для обоих результата.

Если множество~$\underline\gamma$ пусто{\rm,} то возможна ситуация{\rm,} когда ни одна из точек
равновесия из четких стратегий не входит в~$\underline o$, определенное в~\eqref{uo}. Если
считать{\rm,} что при отсутствии кооперации игроки придерживаются точек равновесия из четких
стратегий{\rm,} то кооперация позволит им выбрать вместо точки равновесия точку из~$\underline
o${\rm,} тем самым улучшив по сравнению с~точкой равновесия результат как минимум одного игрока.
Следует заметить{\rm,} что{\rm,} т.\,к. при пустом~$\underline\gamma$ в~$\underline o$ как минимум
два элемента{\rm,} игрокам предстоит договориться о~выборе конкретного элемента из $\underline o$.
В этом случае цель кооперации достигнута.

Если же при пустом~$\underline\gamma$ среди точек равновесия из четких стратегий есть входящие
в~$\underline o${\rm,} то при отсутствии кооперации{\rm,} но присутствии обмена информацией игроки
договорятся о~выборе одной из них{\rm,} предпочитая их точкам равновесия{\rm,} не входящим в~ядро.
Кооперация позволяет сместить выбор на точки $\underline o${\rm,} не являющиеся точками
равновесия{\rm,} но выгодно ли это будет игрокам{\rm,} сказать нельзя.
\end{Theorem}

\section{Выводы}
В статье показано, что в~возможностной модели биматричной игры существуют как игры, в~которых
кооперация улучшает результаты игроков, так и~игры, в~которых кооперация улучшить результаты
игроков не может. Найдены критерии различия таких игр.



\begin{thebibliography}{1}
\bibitem{article}
\BibAuthor{Папилин\;С.\,С., Пытьев\;Ю.\,П.}
 \BibTitle{Вероятностные и~возможностные модели
матричных игр двух субъектов}~// Математическое моделирование.~--- 2010.~--- Т.\,22, \No\,12.~---
С.\,144--160.
    %
\bibitem{book}
 \BibAuthor{Пытьев\;Ю.\,П.}
Возможность как альтернатива вероятности.~---
2-е издание.~---
Москва: Физматлит, 2012.~--- 576~с.
\end{thebibliography}
\end{document}
