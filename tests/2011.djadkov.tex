\documentclass[twoside]{article}
\usepackage{mmro15}
\begin{document}

\title
    [Методы выявления группирования землетрясений]
    {Методы выявления пространственного группирования  землетрясений в~сейсмогеодинамическом исследовании районов Центральной Азии}
\author
    {Дядьков~П.\,Г., Михеева~А.\,В.}
\thanks
    {Работа выполнена при финансовой поддержке РФФИ, проект \No\,10-05-01042-а.}
\email
    {anna@omzg.sscc.ru}
\organization
    {Новосибирск, Институт нефтегазовой геологии и геофизики им.\ А.\,А.~Трофимука Сибирского отделения РАН}
\abstract
    {В~работе описаны методы выявления связанных событий, реализованные в~геоинформационной
системе GIS"=EEDB. Первая группа методов связана с~распознаванием линейных структур
по~распределению событий на~площади с~целью обнаружения приуроченности сейсмичности
к~зонам активных разломов (например, границ плит или блоков). Вторая группа методов
имеет целью выявление афтершоков и~роев, а~после их исключения "--- кластеров взаимосвязанных
землетрясений, плотность распределения которых по~времени не соответствует закону
распределения Пуассона для~случайных величин.}
\maketitle

Изучение закономерностей пространственно"=временного развития сейсмического
процесса в зоне влияния Индо"=Евразийской коллизии в контексте приуроченности
к~основным тектоническим струк\-турным элементам требует разработки методов
кластеризации событий в~пространстве и в про\-ст\-ран\-ст\-ве--вре\-ме\-ни.
Этими методами
был дополнен комплекс геоинформационно"=экспертных методов, реализованный в~вычислительной
интерактивной системе GIS"=EEDB (Geo Information System Expert Earthquake Data
Base), которая была разработана в~ИНГГ СО РАН для проведения работ в~области исследования
сейсмогеодинамического режима.

\section{Описание прикладной системы ана\-лиза данных GIS"=EEDB}
Логическая структура геоинформационной сис\-темы GIS"=EEDB представляет собой совокупность
взаимодействующих между собой программных блоков: сейсмологической базы данных,
географи\-ческой подсистемы и~подсистемы анализа данных.
Поскольку развитие процессов
в~локальной об\-ласти или в~регионе должно рассматриваться с~учетом процессов,
происходящих на~региональном, мегарегиональном и, в~ряде случаев, на~глобальном
уровне, в~сейсмологическую базу данных сис\-те\-мы GIS"=EEDB были включены как глобальные,
так и~региональные каталоги землетрясений.
В~частности, для~пространственно"=временного
исследования  районов Центральной Азии используются региональные каталоги Алтае"=Саянской
об\-ласти, Байкальского региона, Республики Казахстан и~Китая, которые постоянно пополняются
\mbox{новым} материалом.

Исследование, проводимое в системе GIS"=EEDB на различных масштабных уровнях: плит,
микроплит, блоков или вблизи их границ со склад\-чатыми областями, "--- представляет собой
пространственно"=временной анализ ряда параметров сейсмического режима с использованием
алгоритмов математической статистики и анализа временн\'{ы}х рядов.
При этом обычно
изучаются следующие параметры сейсмического режима: сейс\-ми\-че\-ские за\-тишья, наклон
графика пов\-то\-ря\-емости, форшоковая активизация, кластеризация событий, плотность
сейсмогенных разрывов и~т.\,п.
Для \mbox{построения} графиков и карт распределения параметров
сейсмического режима используются характеристики очагов землетрясений выбранного
каталога: сейсмическая энергия, магнитуда, геометрические размеры очага, момент
возникновения землетрясения, координаты эпицентра, глубина очага.
При этом особое
внимание обращается на представительность и достоверность выбранных каталогов
в рас\-смат\-ри\-ва\-емой области.

\section{Выявление блоковых структур}
Для~выявления границ блоков и~других линейных структур по~данным сейсмичности в~системе
GIS"=EEDB реализован простой алгоритм распознавания линейных образов по~множеству
точек, распределенных в~пространстве.
В~основе данного \mbox{метода} лежит задание максимального
шага и~максимального угла отклонения  для~поиска сле\-ду\-ющей точки (эпицентра события).
Угол отклонения рассматривается по~отношению к~нап\-равлению, создаваемому предыдущей парой
соединяемых в~пространстве точек.
Использование этого алгоритма помогает выявлять как
прямолинейные, так и~искривленные (например, дугообразные) структуры (рис.~1).

\begin{figure}[t]
\includegraphics[width=\linewidth]{djadkov1.eps}
\caption{Пример работы алгоритма распознавания линейных структур в~районе
юга Сибири, Монголии и~северного Тибета для~событий с $M\geq4$ (1990--2010): 1 "--- Бусингольская впадина;
2 "--- Главный Саянский \mbox{разлом}; 3 "--- Тункинский разлом; 4 "--- Килианская складчато"=надвиговая зона;
5 "--- Кайдамский блок \cite{lysak}. Использован каталог Китая (CSN), Байкала (БФ ГС СО РАН)
и~Алтая (Алтае"=Саянской экспедиции ГС СО РАН)}
\end{figure}


\section{Методы\;идентификации\;афтершоков}
Большую роль при~проведении исследований, связанных с~вероятностными оценками выборок
событий, играет предварительная обработка исход\-ных наборов данных, в~частности
очищение выбранной части каталога от афтершоков.
Эта задача предполагает создание
алгоритмов выявле\-ния связанных событий, обладающих особыми свойствами пространственно"=временного
распределения.
В~нашей системе реализовано три алгоритма этой операции по~выбору пользователя.
Первый метод, условно названный статистическим, основан на~параметрах
разности времён и~расстояний афтершокового события и~главного толчка, полученных \mbox{нами}
из~статистики накопленных данных об~афтершоковых процессах и~зависящих от~магнитуды
$M_s$ (или энергетического класса~$K$) главного толчка:
%
$$dT=(M_{s\, \text{главн}}-4)162\ \mbox{дней};\quad dS=3L\ \mbox{км},
$$
где $L$ "--- длина очага, определяемая по формуле $\mathrm{lg}L_j=aK_j+c$ \cite{author85}, где $a=0{,}244$,\: $c=-2{,}266$.

Но~наиболее часто используемым в~GIS"=EEDB методом фильтрации афтершоков является
второй метод, названный эллиптическим, основанный на~подходе Прозорова \cite{author86} и~включающий
в~себя следующие этапы:
\begin{enumerate}
%\afterlabel)
\item %1.
Первый проход каталога с~целью нахождения плотности неафтершоковых событий.
\item %2.
Второй проход с~выделением предварительных афтершоков в~прямоугольнике.
\item %3.
Построение по выделенной группе афтершоков эллипса рассеяния по \mbox{среднеквадратичному}
отклонению от центра множества (<<классический>> вариант) или методом наибольшей
вероятности (<<измененный>> вариант) по выбору.
\item %4.
Последующие проходы каталога с~целью послойного выделения афтершоков в~4"=кратной
эллиптической метрике (при выборе <<классического>> варианта).
\end{enumerate}

Время афтершокового процесса определяется как отношение числа афтершоков
к~суммарной плотности в~прямоугольнике или эллипсе. <<Класси\-ческий>> вариант нахождения
параметров эллипса на~этапе 3 соответствует предлагаемому в~\cite{author86} методу расчета метрик
по~среднеквадратичному пространственному откло\-нению точек от~арифметического центра
множества.
При~этом в~системе возможен вариант, учитывающий веса точек, которые
определяются по~числу событий в~ячейке попадания афтершока (рис.~2).
Учёт веса
имеет смысл в~случаях большого разброса афтершокового облака на площади.
Возможен
также <<измененный>> вариант метода, когда метрики определяются с~помощью эллипса
равной вероятности.
Опишем его подробнее.

Расчёт эллипса в этом варианте производится в соответствии с~представлением о~нормальном
распределении случайных величин $x$ и~$y$ относительно центра множества, которое графически
представля\-ется с~помощью эллипсов равной вероятности \cite{author77}:
%
\[
    \phi(x,y) = \frac1{1- \rho^2_{12}} \Bigl(\frac{x^2}{\sigma^2_1}-2 \rho^2_{12} \frac{xy}{\sigma_1 \sigma_2}+\frac{y^2}{\sigma^2_2} \Bigr) = \lambda^2\,,
\]
%
где $\sigma^2_1=DX$, $\sigma^2_2=DY$ "--- дисперсии $x$~и~$y$, а $\rho_{12}$ "--- коэффициент
корреляции между $x$~и~$y$.
В~качестве наилучшей при~рассмотрении вероятностей~$P$,
близких к~$1$, предлагается \cite{author77} аппроксимация квантилей для распределения с~$m=2$
степенями свободы, принимающая при~$P=0{,}9995$ следующий вид:
\begin{multline*}
\lambda^2=m\left(1-\frac2{9m}+u_p\sqrt{\frac2{9m}}\right)^3=\\
=2\left(1-\frac19+3{,}29\frac13\right)^3\,.
\end{multline*}
Результат применения данной оценки для определения эллиптических метрик мы видим
на рис.~2 (эллипс 3).
Поиск афтершоков при этом осуществляется уже в~единственной
крат\-ности (этап 4 отсутствует).

Практика показала, что в ряде случаев преимущество в выявлении афтершоков имеет метод
их идентификации с помощью эллипса равной вероятности.
Так, при выявлении афтершоков
Южнобайкальского землетрясения метод равной вероятности превзошел <<классический>> как
по числу \mbox{выделеннных} событий (861 и 711 соответственно "--- см.\ рис.~2), так и по длительности
афтершоковой последовательности (5,1 и 1,1 года, соответственно "--- рис.~3).
Преимуществом
<<измененного>> метода является также практическая независимость его результатов
от порогового соотношения сигнал/шум $R_{s/n}$~\cite{author86}.

\begin{figure}[t]\centering
\includegraphics[width=0.95\linewidth]{djadkov2.eps}
\caption{Результаты трех вариантов расчёта эллиптических метрик в~алгоритме
выделения афтершоков на~примере землетрясения 25.02.1999 ($M=5{,}9$): 1 "--- по~среднеквадратичному
отклонению без веса; 2 "--- то же с~весом;  3 "--- эллипс равной вероятности.
Цифрами 1 и 2 отмечен внешний эллипс (4"=кратное увеличение метрик расчетного эллипса
согласно методу \cite{author86})}
\end{figure}

\begin{figure}[t]
\includegraphics[width=\linewidth]{djadkov3.eps}
\caption{Распределение по времени афтершоков, выявленных эллипсами 1 и 3;
$R_{s/n}=20$}
\end{figure}

На~примере Чуйского (27.9.2003) землетрясения оба метода выделения афтершоков
(<<классический>> и <<измененный>>) показывают одинаковый результат как в~определении
интервала времени (до~конца детальной части каталога, т.\,е.\ 4,2~года), так~и
по~числу афтершоков (2009 событий).
Это говорит о~сближении качества <<классического>>
и~<<изменен\-ного>> методов в~условиях статистической достаточности по~числу событий.

После процедуры очищения от~афтершоков в~выборке остаётся значительное количество
связанных событий, относящихся к~роевым последо\-вательностям.
Технология очищения
каталогов от роев аналогична удалению афтершоков за исключением условия о~соотношении
магнитуд главного и~зависимых событий "--- в~случае выделения роев зависимые события
могут иметь как меньшую, так и~большую магнитуду по~сравнению с~начальным событием
процесса.
Кроме того, при выявлении роевых последовательностей время процесса не~рассчитывается
(как в \cite{author86} "--- по количеству выделенных афтершоков), а~задается пользователем интерактивно,
поскольку временн\'{о}е распределение событий роя не~обладает свойствами, характерными для
афтершоковых последовательностей.
\begin{figure}[t]\centering
\includegraphics[width=0.95\linewidth]{djadkov4.eps}
\caption{Гистограммы зависимости числа пар соседних по времени событий
с~$M\geq3$ (афтершоки и~рои удалены) в Байкальской рифтовой зоне ($A$) от~$dT$
для периодов: ($B$) "--- с~июля $1987$ по~июнь $1993$; ($C$) "---
с~июля $1993$ по~июнь $1998$; ($D$) "--- с~июля $1998$ по~июнь $2007$.
Шаг суммирования "--- $3$ дня}
\end{figure}

\section{Алгоритмы выявления кластеров}
После удаления афтершоков и роев кластеризация или группирование оставшихся землетрясений
обусловлена существующей в природе ло\-ка\-ли\-зацией сейсмичности в зонах активных \mbox{разломов},
например границ плит или блоков. Для поиска клас\-те\-ров задаются условия на разность
времени и расстояния в каждой паре событий ($dT$ и~$dS$), а также тип кластеризации (временн\'{о}й
или пространственный). В системе GIS"=EEDB заложено два метода нахождения кластеров:

\begin{enumerate*}
%\afterlabel)
\item %1.
метод задания пространственно"=временных интервалов ($dT$~и~$dS$);
\item %2.
метод автоматического расчета $dT$~и~$dS$
исходя из физических процессов разрушения среды \cite{kuksenko,authors85}.
\end{enumerate*}



Важным моментом в первом методе является выбор параметров кластеризации ($dT$~и~$dS$).
Значения задаваемых параметров можно определить по~графикам зависимости числа пар
соседних по времени событий от этих параметров, выявляя интервалы заметного превышения
числа пар относительно графика экспоненциального пуассоновского распределения
(для~$dT$) или максимумов числа событий (для~$dS$).

\balance
Превышение наблюденных значений относительно кривой распределения Пуассона
на графиках ($B$) и ($D$) (рис.~4) в первый 3-дневный интервал может
свидетельствовать о наличии эффекта взаимовлияния сейсмических
событий \cite{authors07} в~эти периоды. Или, например, о~таком аномальном
состоянии среды при ее рассмотрении как динамической системы, при котором
наблюдаются признаки коллективного поведения ее элементов.



\section{Заключение}
Предложены алгоритмы и подходы, по\-зво\-ля\-ющие осуществлять  группирование
гипо\-центров землетрясений в пространственном и пространственно"=временном диапазоне.
Выявление групп связанных событий необходимо как для построения детальных моделей земной
коpы: выделения сейс\-мо\-ак\-тив\-ных границ блоков или отдельных разломов, "---
так и для изучения сейсмического режима территорий.

\medskip
\begin{thebibliography}{1}
\smallskip
\bibitem{lysak}
    \BibAuthor{Лысак\;С.\,В.}
    \BibTitle{Термальная эволюция, геодинамика и современная геотермальная
    активность литосферы Китая}~//
    Геология и геофизика,
    2009, T.~50, \No\,9.~--- С.\,1058--1071.
\bibitem{author85}
    \BibAuthor{Ризниченко\;Ю.\;В.}
    Проблемы сейсмологии. "---
    М.:~Наука, 1985.~--- 408~c.
\bibitem{author86}
    \BibAuthor{Прозоров\;А.\,Г.}
    \BibTitle{Динамический алгоритм выделения афтершоков для мирового каталога
    землетрясений. Математические методы в~сейсмологии и геодинамике}~//
    Вычислительная сейсмология. "---
    М.:~Наука, 1986. "--- Вып. 19.
\bibitem{author77}
    \BibAuthor{Корн\;Г., Корн\;Т.}
    Справочник по математике. %Для научных работников и инженеров.
    "--- М., 1977.~--- 832~c.
\bibitem{kuksenko}
    \BibAuthor{Куксенко\;В.\;С.}
    \BibTitle{Модель перехода от~микро- к~макроразрушению твердых тел}~//
    Физика прочности и пластичности. "---
    Л.:~Наука, 1986.~--- С.\,38--41.
\bibitem{authors85}
    \BibAuthor{Cоболев\;Г.\;А., Пономаpев\;А.\;В.}
    Физика землетpяcений и пpедвеcтники. "---
    М.:~Наука, 2003.~--- 270~c.
\bibitem{authors07}
    \BibAuthor{Ebel\;J.\;E., Kafka\;A.\;L.}
    \BibTitle{A non"=Poissonian element in the seismicity of the Northeastern
    United States}~//
    Bull. Seism. Soc. Amer. "--- 2002. "--- Vol.\,92. No\,5. "--- P.\,2040--2046.
\end{thebibliography}

% Решение Программного Комитета:
%\ACCEPTNOTE
%\AMENDNOTE
%\REJECTNOTE
\end{document}
